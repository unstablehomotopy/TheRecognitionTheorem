\documentclass{book}
\usepackage{fontspec}
\setmainfont{STIX Two Text}

%PACKAGES
\iffalse
Here are the packages that I use
\fi

\usepackage{blindtext, hyperref, verbatim, minted, graphicx, amssymb, textcomp, enumerate, tcolorbox, newunicodechar, textgreek, wasysym, tipa, eso-pic, lipsum, bbold, dsfont}
\usepackage[margin=1.3in]{geometry}
\usepackage{longtable}
\usepackage{newunicodechar}
\usepackage{amsthm}
\usepackage{tikz}
\usepackage{tikz-cd}




%ENVIRONMENTS

%Here I define some common environments. I use definitions, theorems, examples, and lemmas.


\theoremstyle{definition}
\newtheorem{definition}{Definition}
\newtheorem{theorem}{Theorem}
\newtheorem{example}{Example}
\newtheorem{lemma}{Lemma}


\newunicodechar{ₙ}{${}_{n}$}

\newunicodechar{𝓓}{$\mathcal{D}$}
\newunicodechar{∂}{$\partial$}
\newunicodechar{∇}{\raisebox{-0.05cm}{$\nabla$}}

%\newunicodechar{Π⃗}{$\stackrel{\arr}{\pi}$}

\newunicodechar{×}{$\times$}
\newunicodechar{→}{$\rightarrow$}
\newunicodechar{⟨}{$\langle$}
\newunicodechar{⟩}{$\rangle$}
\newunicodechar{↦}{$\mapsto$}
\newunicodechar{∧}{$\wedge$}
\newunicodechar{∨}{$\vee$}
\newunicodechar{∃}{$\exists$}
\newunicodechar{∀}{$\forall$}
\newunicodechar{¬}{$\neg$}
\newunicodechar{ᵃ}{${}^{\texttt{a}}$}
\newunicodechar{ᵇ}{${}^{\texttt{b}}$}
\newunicodechar{ᶜ}{${}^{\texttt{c}}$}
\newunicodechar{ᵈ}{${}^{\texttt{d}}$}
\newunicodechar{ᵉ}{${}^{\texttt{e}}$}
\newunicodechar{ᶠ}{${}^{\texttt{f}}$}
\newunicodechar{ᵍ}{${}^{\texttt{g}}$}
\newunicodechar{ʰ}{${}^{\texttt{h}}$}
\newunicodechar{ⁱ}{${}^{\texttt{i}}$}
\newunicodechar{ʲ}{${}^{\texttt{j}}$}
\newunicodechar{ᵏ}{${}^{\texttt{k}}$}
\newunicodechar{ˡ}{${}^{\texttt{l}}$}
\newunicodechar{ᵐ}{${}^{\texttt{m}}$}
\newunicodechar{ⁿ}{${}^{\texttt{n}}$}
\newunicodechar{ᵒ}{${}^{\texttt{o}}$}
\newunicodechar{ᵖ}{${}^{\texttt{p}}$}
\newunicodechar{ʳ}{${}^{\texttt{r}}$}
\newunicodechar{ˢ}{${}^{\texttt{s}}$}
\newunicodechar{ᵗ}{${}^{\texttt{t}}$}
\newunicodechar{ᵘ}{${}^{\texttt{u}}$}
\newunicodechar{ᵛ}{${}^{\texttt{v}}$}
\newunicodechar{ʷ}{${}^{\texttt{w}}$}
\newunicodechar{ˣ}{${}^{\texttt{x}}$}
\newunicodechar{ʸ}{${}^{\texttt{y}}$}
\newunicodechar{ᶻ}{${}^{\texttt{z}}$}
\newunicodechar{⁰}{${}^{\texttt{0}}$}
\newunicodechar{¹}{${}^{\texttt{1}}$}
\newunicodechar{²}{${}^{\texttt{2}}$}
\newunicodechar{³}{${}^{\texttt{3}}$}
\newunicodechar{⁴}{${}^{\texttt{4}}$}
\newunicodechar{⁵}{${}^{\texttt{5}}$}
\newunicodechar{⁶}{${}^{\texttt{6}}$}
\newunicodechar{⁷}{${}^{\texttt{7}}$}
\newunicodechar{⁸}{${}^{\texttt{8}}$}
\newunicodechar{⁹}{${}^{\texttt{9}}$}
\newunicodechar{⁻}{${}^{\texttt{-}}$}
\newunicodechar{ᵒ}{${}^{\texttt{o}}$}
\newunicodechar{ᵖ}{${}^{\texttt{ω}}$}
\newunicodechar{⁻}{${}^{\texttt{-}}$}
\newunicodechar{¹}{${}^{\texttt{1}}$}
\newunicodechar{₀}{${}_{\texttt{0}}$}
\newunicodechar{₁}{${}_{\texttt{1}}$}
\newunicodechar{₂}{${}_{\texttt{2}}$}
\newunicodechar{₃}{${}_{\texttt{3}}$}
\newunicodechar{₄}{${}_{\texttt{4}}$}
\newunicodechar{₅}{${}_{\texttt{5}}$}
\newunicodechar{₆}{${}_{\texttt{6}}$}
\newunicodechar{₇}{${}_{\texttt{7}}$}
\newunicodechar{₈}{${}_{\texttt{8}}$}
\newunicodechar{₉}{${}_{\texttt{9}}$}
\newunicodechar{𝔸}{$\mathbb{A}$}
\newunicodechar{𝔹}{$\mathbb{B}$}
\newunicodechar{ℂ}{$\mathbb{C}$}
\newunicodechar{𝔻}{$\mathbb{D}$}
\newunicodechar{𝔼}{$\mathbb{E}$}
\newunicodechar{𝔽}{$\mathbb{F}$}
\newunicodechar{𝔾}{$\mathbb{G}$}
\newunicodechar{ℍ}{$\mathbb{H}$}
\newunicodechar{𝕀}{$\mathbb{I}$}
\newunicodechar{𝕁}{$\mathbb{J}$}
\newunicodechar{𝕂}{$\mathbb{K}$}
\newunicodechar{𝕃}{$\mathbb{L}$}
\newunicodechar{𝕄}{$\mathbb{M}$}
\newunicodechar{ℕ}{$\mathbb{N}$} 
\newunicodechar{𝕆}{$\mathbb{O}$}
\newunicodechar{ℙ}{$\mathbb{P}$}
\newunicodechar{ℚ}{$\mathbb{Q}$}
\newunicodechar{ℝ}{$\mathbb{R}$}
\newunicodechar{𝕊}{$\mathbb{S}$}
\newunicodechar{𝕋}{$\mathbb{T}$} 
\newunicodechar{𝕌}{$\mathbb{U}$}
\newunicodechar{𝕍}{$\mathbb{V}$}
\newunicodechar{𝕎}{$\mathbb{W}$}
\newunicodechar{𝕏}{$\mathbb{X}$}
\newunicodechar{𝕐}{$\mathbb{Y}$}
\newunicodechar{ℤ}{$\mathbb{Z}$}
\newunicodechar{𝕒}{$\mathbb{a}$}
\newunicodechar{𝕓}{$\mathbb{b}$}
\newunicodechar{𝕔}{$\mathbb{c}$}
\newunicodechar{𝕕}{$\mathbb{d}$}
\newunicodechar{𝕖}{$\mathbb{e}$}
\newunicodechar{𝕗}{$\mathbb{f}$}
\newunicodechar{𝕘}{$\mathbb{g}$}
\newunicodechar{𝕙}{$\mathbb{h}$}
\newunicodechar{𝕚}{$\mathbb{i}$}
\newunicodechar{𝕛}{$\mathbb{j}$}
\newunicodechar{𝕜}{$\mathbb{k}$}%𝔸𝔹ℂ𝔻𝔼𝔽𝔾ℍ𝕀𝕁𝕂𝕃𝕄ℕ𝕆ℙℚℝ𝕊𝕋𝕌𝕍𝕎𝕏𝕐ℤ𝕒𝕓𝕔𝕕𝕖𝕗𝕘𝕙𝕚𝕛𝕜𝕝𝕞𝕟𝕠𝕡𝕢𝕣𝕤𝕥𝕦𝕧𝕨𝕩𝕪𝕫
\newunicodechar{𝕝}{$\mathbb{l}$} 
\newunicodechar{𝕞}{$\mathbb{m}$}
\newunicodechar{𝕟}{$\mathbb{n}$}
\newunicodechar{𝕠}{$\mathbb{o}$}
\newunicodechar{𝕡}{$\mathbb{p}$}
\newunicodechar{𝕢}{$\mathbb{q}$}
\newunicodechar{𝕣}{$\mathbb{r}$}
\newunicodechar{𝕤}{$\mathbb{s}$}
\newunicodechar{𝕥}{$\mathbb{t}$}
\newunicodechar{𝕦}{$\mathbb{u}$}
\newunicodechar{𝕧}{$\mathbb{v}$}
\newunicodechar{𝕨}{$\mathbb{w}$}
\newunicodechar{𝕩}{$\mathbb{x}$}
\newunicodechar{𝕪}{$\mathbb{y}$}
\newunicodechar{𝕫}{$\mathbb{z}$}
\newunicodechar{𝚫}{$\Delta$}
\newunicodechar{ʃ}{$\int$}
\newunicodechar{∪}{$\cup$}
\newunicodechar{∩}{$\cap$}
\newunicodechar{±}{$\pm$}
\newunicodechar{𝔄}{$\mathfrak{A}$}




\newunicodechar{𝔅}{$\mathfrak{B}$}
\newunicodechar{ℭ}{$\mathfrak{C}$}
\newunicodechar{𝔇}{$\mathfrak{D}$}
\newunicodechar{𝔈}{$\mathfrak{E}$}
\newunicodechar{𝔉}{$\mathfrak{F}$}
\newunicodechar{𝔊}{$\mathfrak{G}$}
\newunicodechar{ℌ}{$\mathfrak{H}$}
\newunicodechar{ℑ}{$\mathfrak{I}$}
\newunicodechar{𝔍}{$\mathfrak{J}$}
\newunicodechar{𝔎}{$\mathfrak{K}$}
\newunicodechar{𝔏}{$\mathfrak{L}$}
\newunicodechar{𝔐}{$\mathfrak{M}$}
\newunicodechar{𝔑}{$\mathfrak{N}$}
\newunicodechar{𝔒}{$\mathfrak{O}$}
\newunicodechar{𝔓}{$\mathfrak{P}$}
\newunicodechar{𝔔}{$\mathfrak{Q}$}
\newunicodechar{ℜ}{$\mathfrak{R}$}
\newunicodechar{𝔖}{$\mathfrak{S}$}
\newunicodechar{𝔗}{$\mathfrak{T}$}
\newunicodechar{𝔘}{$\mathfrak{U}$}
\newunicodechar{𝔙}{$\mathfrak{V}$}
\newunicodechar{𝔚}{$\mathfrak{W}$}
\newunicodechar{𝔛}{$\mathfrak{X}$}
\newunicodechar{𝔜}{$\mathfrak{Y}$}
\newunicodechar{ℨ}{$\mathfrak{Z}$}

\newunicodechar{𝔞}{$\mathfrak{a}$}
\newunicodechar{𝔟}{$\mathfrak b$}
\newunicodechar{𝔠}{$\mathfrak{c}$}
\newunicodechar{𝔡}{$\mathfrak{d}$}
\newunicodechar{𝔢}{$\mathfrak{e}$}
\newunicodechar{𝔣}{$\mathfrak{f}$}
\newunicodechar{𝔤}{$\mathfrak{g}$}
\newunicodechar{𝔥}{$\mathfrak{h}$}
\newunicodechar{𝔦}{$\mathfrak{i}$}
\newunicodechar{𝔧}{$\mathfrak{j}$}
\newunicodechar{𝔨}{$\mathfrak{k}$}
\newunicodechar{𝔩}{$\mathfrak{l}$}
\newunicodechar{𝔪}{$\mathfrak{m}$}
\newunicodechar{𝔫}{$\mathfrak{n}$}
\newunicodechar{𝔬}{$\mathfrak{o}$}
\newunicodechar{𝔭}{$\mathfrak{ω}$}
\newunicodechar{𝔮}{$\mathfrak{q}$}
\newunicodechar{𝔯}{$\mathfrak{r}$}
\newunicodechar{𝔰}{$\mathfrak{s}$}
\newunicodechar{𝔱}{$\mathfrak{t}$}
\newunicodechar{𝔲}{$\mathfrak{u}$}
\newunicodechar{𝔳}{$\mathfrak{v}$}
\newunicodechar{𝔴}{$\mathfrak{w}$}
\newunicodechar{𝔵}{$\mathfrak{x}$}
\newunicodechar{𝔶}{$\mathfrak{y}$}
\newunicodechar{𝔷}{$\mathfrak{z}$}

\newunicodechar{𝐀}{${\bf{A}}$}
\newunicodechar{𝐁}{${\bf{B}}$}
\newunicodechar{𝐂}{${\bf{C}}$}
\newunicodechar{𝐃}{${\bf{D}}$}
\newunicodechar{𝐄}{${\bf{E}}$}
\newunicodechar{𝐅}{${\bf{F}}$}
\newunicodechar{𝐆}{${\bf{G}}$}
\newunicodechar{𝐇}{${\bf{H}}$}
\newunicodechar{𝐈}{${\bf{I}}$}
\newunicodechar{𝐉}{${\bf{J}}$}
\newunicodechar{𝐊}{${\bf{K}}$}
\newunicodechar{𝐋}{${\bf{L}}$}
\newunicodechar{𝐌}{${\bf{M}}$}
\newunicodechar{𝐍}{${\bf{N}}$}
\newunicodechar{𝐎}{${\bf{O}}$}
\newunicodechar{𝐏}{${\bf{P}}$}
\newunicodechar{𝐐}{${\bf{Q}}$}
\newunicodechar{𝐑}{${\bf{R}}$}
\newunicodechar{𝐒}{${\bf{S}}$}
\newunicodechar{𝐓}{${\bf{T}}$}
\newunicodechar{𝐔}{${\bf{U}}$}
\newunicodechar{𝐕}{${\bf{V}}$}
\newunicodechar{𝐖}{${\bf{W}}$}
\newunicodechar{𝐗}{${\bf{X}}$}
\newunicodechar{𝐘}{${\bf{Y}}$}
\newunicodechar{𝐙}{${\bf{Z}}$}

\newunicodechar{𝐚}{${\bf{a}}$}
\newunicodechar{𝐛}{${\bf{b}}$}
\newunicodechar{𝐜}{${\bf{c}}$}
\newunicodechar{𝐝}{${\bf{d}}$}
\newunicodechar{𝐞}{${\bf{e}}$}
\newunicodechar{𝐟}{${\bf{f}}$}
\newunicodechar{𝐠}{${\bf{g}}$}
\newunicodechar{𝐡}{${\bf{h}}$}
\newunicodechar{𝐢}{${\bf{i}}$}
\newunicodechar{𝐣}{${\bf{j}}$}
\newunicodechar{𝐤}{${\bf{k}}$}
\newunicodechar{𝐥}{${\bf{l}}$}
\newunicodechar{𝐦}{${\bf{m}}$}
\newunicodechar{𝐧}{${\bf{n}}$}
\newunicodechar{𝐨}{${\bf{o}}$}
\newunicodechar{𝐩}{${\bf{ω}}$}
\newunicodechar{𝐪}{${\bf{q}}$}
\newunicodechar{𝐫}{${\bf{r}}$}
\newunicodechar{𝐬}{${\bf{s}}$}
\newunicodechar{𝐭}{${\bf{t}}$}
\newunicodechar{𝐮}{${\bf{u}}$}
\newunicodechar{𝐯}{${\bf{v}}$}
\newunicodechar{𝐰}{${\bf{w}}$}
\newunicodechar{𝐱}{${\bf{x}}$}
\newunicodechar{𝐲}{${\bf{y}}$}
\newunicodechar{𝐳}{${\bf{z}}$}

\newunicodechar{⊣}{\ensuremath{\dashv}}
\newunicodechar{ॱ}{${}^{\cdot}$}
\newunicodechar{𛲔}{${}_{\cdot}$}
\newunicodechar{⋯}{$\cdots$}
\newunicodechar{⇄}{$\rightleftarrows$}
\newunicodechar{⇆}{$\leftrightarrows$}

\newunicodechar{ꜝ}{$\raisebox{1ex}{\scalebox{0.5}{\texttt{!}}}$}
\newunicodechar{ꜞ}{$\raisebox{1ex}{\scalebox{0.5}{\texttt{¡}}}$}



%This is notation we will use for categories


\newunicodechar{𝟙}{$\mathbb{1}$}
\newunicodechar{∘}{$\circ$}

%This is notation we will use for twocategories


\newunicodechar{𝟏}{${\bold{1}}$}
\newunicodechar{⭢}{$\longrightarrow$}
\newunicodechar{•}{${\bullet}$}
\newunicodechar{∙}{${\bullet}$}

%This is notation we will use for ∞-ℂ𝕒𝕥

\newunicodechar{よ}{$\includegraphics[width=0.27cm,height=0.27cm]{yon.png}$}
\newunicodechar{⊥}{$\bot$}
\newunicodechar{∼}{$\sim$}
\newunicodechar{≃}{$\simeq$}
\newunicodechar{≅}{$\cong$}
\newunicodechar{∞}{$\infty$}

\newunicodechar{α}{$\alpha$}
\newunicodechar{β}{$\beta$}
\newunicodechar{γ}{$\gamma$}
\newunicodechar{δ}{$\delta$}
\newunicodechar{ε}{$\epsilon$}
\newunicodechar{η}{$\eta$}
\newunicodechar{ζ}{$\zeta$}
\newunicodechar{θ}{$\theta$}
\newunicodechar{ι}{$\iota$}
\newunicodechar{μ}{$\mu$}
\newunicodechar{κ}{$\kappa$}
\newunicodechar{λ}{$\lambda$}
\newunicodechar{ρ}{$\rho$}
\newunicodechar{π}{$\pi$}
\newunicodechar{σ}{$\sigma$}
\newunicodechar{τ}{$\tau$}
\newunicodechar{υ}{$\upsilon$}
\newunicodechar{φ}{$\phi$}
\newunicodechar{ψ}{$\psi$}
\newunicodechar{ξ}{$\xi$}
\newunicodechar{χ}{$\chi$}
\newunicodechar{ω}{$\omega$}

\newunicodechar{⊗}{$\otimes$}

\makeatletter
\newcommand*{\shifttext}[2]{\settowidth{\@tempdima}{#2}\makebox[\@tempdima]{\hspace*{#1}#2}}
\makeatother
\definecolor{Red}{cmyk}{0.1, 0.70, 0.65, 0.00, 1.00}
\definecolor{Blue}{cmyk}{0.9, 0.2, 0.2, 0.00, 1.00}
\definecolor{Yellow}{cmyk}{0.0, 0.00, 0.7, 0.00, 0.5}
\definecolor{Green}{cmyk}{0.6, 0.0, 0.6, 0.00, 1.00}
\definecolor{Purple}{cmyk}{0.8, 0.3, 0.3, 0.00, 1.00}
\definecolor{Orange}{cmyk}{0.0, 0.3, 0.7, 0.00, 1.00}
\definecolor{Grey}{cmyk}{0.13, 0.13, 0.13, 0.00, 1.00}
\newcounter{definitioncounter}
\setcounter{definitioncounter}{1}
\newcounter{theoremcounter}
\setcounter{theoremcounter}{1}
\newcounter{printcounter}
\setcounter{printcounter}{1}
\newcounter{examplecounter}
\setcounter{examplecounter}{1}
\newcounter{ccounter}
\setcounter{ccounter}{1}
\newcounter{pcounter}
\setcounter{pcounter}{1}
\newcounter{lcounter}
\setcounter{lcounter}{1}
\newcounter{sectioncount}
\newcounter{subsectioncount}
\setcounter{sectioncount}{1}
\renewcommand{\section}[1]{\newpage\ \\ \ \\ \begin{center} \scalebox{1.5}{\texttt{\thesectioncount . #1}} \stepcounter{sectioncount} \setcounter{subsectioncount}{1} \end{center} \begin{center} \ \\ \ \\ \thispagestyle{empty} \end{center}}
\renewcommand{\subsection}[1]{\texttt{\thesubsectioncount . #1} \stepcounter{subsectioncount}}
\renewcommand{\backslash}{\reflectbox{\texttt{/}}}

\newcounter{chaptercount}
\renewcommand{\chapter}[1]{
\newpage
{
\Huge 
\begin{center}
\ \\
\ \\
\thispagestyle{empty}
\texttt{#1}
\end{center}}
\ \\
\ \\
}

\newcounter{partcount}
\stepcounter{partcount}
\renewcommand{\part}[1]{
\newpage
{
\Huge 
\begin{center}
\ \\
\ \\
\ \\
\ \\
\ \\
\ \\
\thispagestyle{empty}
\texttt{PART {\thepartcount}: #1}
\stepcounter{partcount}
\end{center}}
\ \\
\ \\
}


\begin{document}

\thispagestyle{empty} 

\AddToShipoutPicture*
    {\put(540,720){

    \href{http://www.linearlibrary.net}{\includegraphics[width=2cm,height=2cm]{ll.png}}

    }}

\AddToShipoutPicture*
  {\put(470,767){
    \href{https://github.com/linlib/CategoriesandHilbertSpaces/StringDiagramGenerator.py}{\texttt{.py file}}
  }}

\AddToShipoutPicture*
  {\put(470,752){
    \href{https://github.com/linlib/TheWhiteheadTheoremandTwoVariations/blob/main/TheWhiteheadTheoremandTwoVariations.tex}{\texttt{.tex file}}\\

  }}


\AddToShipoutPicture*
  {\put(470,737){

    \href{https://github.com/linlib/TheWhiteheadTheoremandTwoVariations/blob/main/TheWhiteheadTheoremandTwoVariations.pdf}{\texttt{.pdf file}}\\

  }}

  \AddToShipoutPicture*
  {\put(470,722){
    \href{https://github.com/linlib/TheWhiteheadTheoremandTwoVariations/blob/main/TheWhiteheadTheoremandTwoVariations.lean}{\texttt{.lean file}}

  }}

\ \\

%LEAN: 
\begin{center}
\begin{tcolorbox}[width=5.8in,colback={white},coltitle=white]
\begin{center}
\ \\
\scalebox{3}{\texttt{The Whitehead Theorem}}\\
\ \\
\scalebox{3}{\texttt{and}}\\
\ \\
\scalebox{3}{\texttt{Two Variations}}\\
\ \\
\end{center}
\end{tcolorbox}
\end{center}
\ \\



\ \\
\begin{center}
\texttt{Plans to prove the Whitehead theorem}\\
\texttt{in Lean 4, with extensive use of Mathlib 4}
\end{center}
\ \\
\ \\


\thispagestyle{empty}


\newpage

\ \\
\ \\

\begin{center}

\pagecolor{white}
\color{black}

\end{center}

\thispagestyle{empty}




\newpage
\pagecolor{white}
\color{black}
\ \\
\ \\
\thispagestyle{empty}
\large %%%%%%%% HERE IS THE large LARGE size textsize set text size
\newpage 
\ \\
\ \\
\ \\
\ \\
\ \\
\ \\
\ \\
\ \\
\ \\
\ \\
\ \\
\thispagestyle{empty}
 
\newpage


\ \\
\ \\
\ \\
\ \\
\ \\
\ \\
\ \\
\ \\
\ \\
\ \\
\ \\

We wish to acknowledge the collaborative efforts of E. Dean Young and Jiazhen Xia. Dean Young initially formulated the introduction with twelve goals, posting them on the Lean Zulip in August of 2023. Together the authors are pursuing these plans as a long term project.\\



\newpage



\newpage
\section{Contents}

The table of contents below reflects the tentative long-term goals of the authors, with the main goal the pursuit of the Whitehead theorem for a point-set model involving Mathlib's predefined homotopy groups.\\


\iffalse
{
\footnotesize
\begin{longtable}{|| l || l ||} 
\hline
\multicolumn{1}{||c||}{$\texttt{Section}$} & \multicolumn{1}{|c||}{$\texttt{Description}$} \\
\hline
\hline
 &  \\
\hline
 &  \\
\hline
 &  \\
\hline
 &  \\
\hline \hline
\multicolumn{2}{||c||}{\texttt{PART I: } ...} \\
\hline \hline
\multicolumn{2}{||c||}{\texttt{Chapter 1: } ...} \\
\hline \hline
\hline
\multicolumn{2}{||c||}{\texttt{Chapter 3: }...} \\
\hline \hline
 & \\
\hline
 & \\
\multicolumn{2}{||c||}{\texttt{Chapter 4: }...} \\
\hline \hline
 & \\
\hline
 & \\
\hline \hline
\multicolumn{2}{||c||}{\texttt{PART II: } BASED CONNECTED ∞-CATEGORIES} \\
\hline \hline
\multicolumn{2}{||c||}{\texttt{Chapter 5: }...} \\
\hline \hline
OperadicMonoids & \\
\hline 
OperadicMonoidActions & \\
\hline \hline
\multicolumn{2}{||c||}{\texttt{Chapter 6: }...} \\
\hline \hline
 & \\
\hline
  & \\
\hline
  & \\
\hline
  & \\
 \hline
  & \\
\hline
  & \\
\hline \hline
\multicolumn{2}{||c||}{\texttt{Chapter 7: }...} \\
\hline \hline
 & \\
\hline
 & \\
\multicolumn{2}{||c||}{\texttt{Chapter 8: }...} \\
\hline \hline
 & \\
\hline
 & \\
\hline \hline
\multicolumn{2}{||c||}{\texttt{PART III: } ∞-GROUPOIDS} \\
\hline \hline
\multicolumn{2}{||c||}{\texttt{Chapter 9: }...} \\
\hline \hline
OperadicGroupoids & \\
\hline 
OperadicGroupoidActions & \\
\hline \hline
\multicolumn{2}{||c||}{\texttt{Chapter 10: }...} \\
\hline \hline
E⃡ & \\
\hline
e⃡  & \\
\hline
B⃡  & \\
\hline
b⃡  & \\
 \hline
∂⃡  & \\
\hline
∇⃡ & \\
\hline \hline
\multicolumn{2}{||c||}{\texttt{Chapter 11: }...} \\
\hline \hline
 & \\
\hline
 & \\
\multicolumn{2}{||c||}{\texttt{Chapter 12: }...} \\
\hline \hline
 & \\
\hline
 & \\
\hline \hline
\multicolumn{2}{||c||}{\texttt{PART III: } ∞-CATEGORIES} \\
\hline \hline
\multicolumn{2}{||c||}{\texttt{Chapter 13: }...} \\
\hline \hline
OperadicCategory & \\
\hline
OperadicPresheaves & \\
\hline \hline
\multicolumn{2}{||c||}{\texttt{Chapter 14: }...} \\
\hline \hline
E⃗  & \\ 
\hline
e⃗ & \\
\hline
B⃗ & \\
\hline
b⃗ & \\
 \hline
∂⃗ & \\
\hline
∇⃗ & \\
\hline \hline
\multicolumn{2}{||c||}{\texttt{Chapter 15: }The Recognition Theorem for ∞-Categories} \\
\hline \hline
 & \\
\hline
 & \\
\hline \hline
\multicolumn{2}{||c||}{\texttt{Chapter 16: }The Classifying Space Theorem for ∞-Categories} \\
\hline \hline
 & \\
\hline
 & \\
\hline \hline
\end{longtable}
}
\fi

\noindent\textcolor{Red}{\rule{16cm}{1mm}}
\begin{center}
\texttt{Implementation Progress}
\end{center}
\noindent\textcolor{Red}{\rule{16cm}{1mm}}

\noindent\textcolor{Red}{\rule{16cm}{1mm}}
\begin{center}
\texttt{Writing Progress}
\end{center}
\noindent\textcolor{Red}{\rule{16cm}{1mm}}

In ```TheWhiteheadTheoremandTwoVariations''', we will define six "internal" structures based on the ones found in ```Galois Theories''' by Janelidze and Borceux, as well as six ```operadic''' structures.

\iffalse
Ω⃗, Ω⃡, Ω⃗₀, Ω⃡₀ are the ones I construct and Ω Mathlib's.
\fi



\newpage
\section{Contents}





\newpage 
\ \\
\ \\
\ \\
\ \\
\ \\
\ \\
%LEAN: 
\begin{center}
\begin{tcolorbox}[width=5in,colback={white},title={\begin{center}\texttt{About the Author} \addtocounter{lcounter}{1}  \end{center}},colbacktitle=Yellow,coltitle=black]
Dean Young is a master's student at New York University, where he studies mathematics. \\
\begin{center}
\includegraphics[width=7.5cm,height=5cm]{about.jpg}
\end{center}
\end{tcolorbox}
\end{center}

\begin{center}
\begin{tcolorbox}[width=5in,colback={white},title={\begin{center}\texttt{About the Author} \addtocounter{lcounter}{1}  \end{center}},colbacktitle=Yellow,coltitle=black]
Jiazhen Xia is a graduate student at Zhejiang University, where he studies computer science. \\
\begin{center}
\includegraphics[width=7.5cm]{about2.jpg}
\end{center}
\end{tcolorbox}
\end{center}
\newpage
\ \\
\thispagestyle{empty}






\end{document}







































